\documentclass{beerydoc}


\ExplSyntaxOn
\usepackage[overwrite-date]{clockdate}
\clockdatesetup
  {
      date / star =
        {
            days-final-separator   = { ,~or~ }
          , days-pair-separator    = {  ~or~ }
          , months-final-separator = { ,~or~ }
          , months-pair-separator  = {  ~or~ }
        }
    , date / month = long
    , date / order = day-month-year
  }
\ExplSyntaxOff


\begin{document}


\section*{\Large The \pkg{clockdate} package}

Print the time of day and the calendar date.

Oliver Beery

Version 0.0.0\quad\date{2026-01-10}


\section{Introduction}
\label{sec:introduction}

\subsection{About}

This package provides \cs{clock} and \cs{calendardate}.
\cs{clock} prints the time of day.
\cs{calendardate} prints the calendar date.

\subsection{Loading the package}
\label{subsec:loading}

\listheading{Requirements:}
\begin{itemize}
  \item \LaTeXe{} version 2025-11-01 or newer
  \item \pkg{l3kernel} version 2025-10-09 or newer
\end{itemize}

\subsection{Syntax}
\label{subsec:syntax}

This documentation uses the syntax \meta{integer}.
This syntax means any integer as understood by \TeX{} such as an explicit
integer, a count register, or a macro that expands to an integer.


\section{Package option}
\label{sec:packageoption}

\begin{code}{key:overwrite-date}
  \begin{syntax}
    \key{overwrite-date}
  \end{syntax}
  Makes \cs{date} equivalent to \cs{calendardate}, overwriting its previous
  definition.
  This is done at the \hook{begindocument} hook.
  By default, \cs{date} sets the document\rq s date which is used in
  \cs{maketitle}.
\end{code}


\section{Commands}
\label{sec:commands}

\begin{code}{\clockdatesetup}
  \begin{syntax}
    \cs{clockdatesetup} \sarg{} \marg{key-value list}
  \end{syntax}
  Sets and processes the \pkg{clockdate} package keys (\S\ref{sec:keys}) in
  \meta{key-value list}.
  Adding the optional star first sets all the \pkg{clockdate} package keys to
  their initial values.
  All assignments made by \cs{clockdatesetup} are local to the current group.
  Can be used mid-document.
\end{code}

\begin{code}{\clock}
  \begin{syntax}
    \cs{clock} \sarg{} \oarg{key-value list} \marg{token list}
  \end{syntax}
  Prints the time of day or a time of day range.
  \meta{token list} must take either of the following forms:
  \begin{itemize}
    \item \meta{clock}
    \item \meta{clock_1}|--|\meta{clock_2}
  \end{itemize}

  \meta{clock} must take any of the following forms:
  \begin{itemize}
    \item \meta{hour}|:|\meta{minute}|:|\meta{second}
    \item \meta{hour}|:|\meta{minute}
    \item \meta{hour}
  \end{itemize}

  \meta{hour}, \meta{minute}, and \meta{second} must be an \meta{integer}.
  \meta{hour} must be an \meta{integer} from \num{0} to \num{24}.
  \meta{minute} and \meta{second} must be an \meta{integer} from \num{0} to
  \num{59}.

  If the optional argument is used, the keys in \meta{key-value list} are set
  in path |clock| for only that particular usage of \cs{clock}.
  For details on adding the optional star, see the key \key{clock/star}
  (\S\ref{subsec:clock}).
\end{code}

\begin{code}{\calendardate}
  \begin{syntax}
    \cs{calendardate} \sarg{} \oarg{key-value list} \marg{token list}
  \end{syntax}
  \listheading
    {
      Prints the calendar date.
      \meta{token list} must take any of the following forms:
    }
  \begin{itemize}
    \item \meta{year}|-|\meta{month}|-|\meta{day(s)}
    \item \meta{month}|-|\meta{day(s)}
    \item \meta{year}|-|\meta{month(s)}
  \end{itemize}

  \meta{year} and \meta{month} must be an \meta{integer}.
  \meta{year} must be an \meta{integer} from \num{1000} to \num{9999}.
  \meta{month} must be an \meta{integer} from \num{1} to \num{12}.

  \listheading{\meta{day(s)} must take any of the following forms:}
  \begin{itemize}
    \item a single day \meta{day}
    \item a day range \meta{day_1}|--|\meta{day_2}
    \item a comma-separated list of items where each item must be either a
    single day \meta{day} or a day range \meta{day_1}|--|\meta{day_2}
  \end{itemize}
  \meta{day} must be an \meta{integer}.
  The calendar date must be valid.
  \date{2-29} is always valid if \meta{year} is omitted.

  \listheading{\meta{month(s)} must take any of the following forms:}
  \begin{itemize}
    \item a single month \meta{month}
    \item a month range \meta{month_1}|--|\meta{month_2}
    \item a comma-separated list of items where each item must be either a
    single month \meta{month} or a month range \meta{month_1}|--|\meta{month_2}
  \end{itemize}

  If the optional argument is used, the keys in \meta{key-value list} are set
  in path |date| for only that particular usage of \cs{calendardate}.
  For details on adding the optional star, see the key \key{date/star}
  (\S\ref{subsec:date}).
\end{code}


\section{Keys}
\label{sec:keys}

This section documents the keys provided by the \pkg{clockdate} package.
Set the package keys using \cs{clockdatesetup} (\S\ref{sec:commands}).


\subsection{\cs{clock}}
\label{subsec:clock}

This subsection documents the keys that modify the behavior of \cs{clock}.

\begin{code}{key:clock}
  \begin{syntax}
    \key{clock=\meta{key-value list}}
  \end{syntax}
  Meta key that sets the keys in \meta{key-value list} in path |clock|.
\end{code}

\begin{code}{key:clock/star}
  \begin{syntax}
    \key{clock/star=\meta{key-value list}}
  \end{syntax}
  When adding the optional star in \cs{clock}, the keys in \meta{key-value
  list} in path |clock| will be set for only that particular usage of
  \cs{clock}.
  The initial value is \meta{empty}.
\end{code}

\begin{code}{key:clock/hour}
  \begin{syntax}
    \key{clock/hour=\choices{12,24}}
  \end{syntax}
  Choice key that sets whether to print the hour in 12-hour or 24-hour format.
  In 24-hour format, the am/pm is always omitted.
  The initial value is \key{=12}.
\end{code}

\begin{code}{key:clock/hour-leading-zero}
  \begin{syntax}
    \key{clock/hour-leading-zero=\choices{true,false}}
  \end{syntax}
  Boolean key that sets whether to print the hour with a leading zero.
  The initial value is \key{=false}.
\end{code}

\begin{code}{key:clock/range-separator}
  \begin{syntax}
    \key{clock/range-separator=\meta{token list}}
  \end{syntax}
  Sets the separator between \meta{clock_1} and \meta{clock_2} in a time of day
  range to \meta{token list}.
  The initial value is \key{= to }.
\end{code}

\subsubsection{Printing the am/pm}

\begin{code}{key:clock/ampm-omit-first}
  \begin{syntax}
    \key{clock/ampm-omit-first=\choices{true,false}}
  \end{syntax}
  Boolean key that sets whether \meta{clock_1} prints the am/pm in a time of
  day range if \meta{clock_1} and \meta{clock_2} would either both display am
  or both display pm.
  The initial value is \key{=true}.
\end{code}

\begin{code}{key:clock/ampm-separator}
  \begin{syntax}
    \key{clock/ampm-separator=\meta{token list}}
  \end{syntax}
  Sets the separator before the am/pm to \meta{token list}.
  The initial value is \key{= }.
\end{code}

\begin{code}{key:clock/ampm}
  \begin{syntax}
    \key{clock/ampm=\meta{choice}}
  \end{syntax}
  \listheading
    {
      Choice key that sets the am/pm format.
      \meta{choice} must match any of the following:
    }
  \begin{itemize}
    \item \key{=lowercase-with-periods} (a.m./p.m.)
    \item \key{=lowercase} (am/pm)
    \item \key{=uppercase-with-periods} (A.M./P.M.)
    \item \key{=uppercase} (AM/PM)
    \item
    \key{=small-caps-with-periods}
    (\textsc{a}.\textsc{m}./\textsc{p}.\textsc{m}.)
    \item \key{=small-caps} (\textsc{am}/\textsc{pm})
    \item \key{=none}
  \end{itemize}
  The initial value is \key{=lowercase-with-periods}.

  If the next token after \cs{clock} is a period, the am/pm will not print an
  extra period.
  When the am/pm prints the period, it is followed by \cs{@} because the
  sentence does not end here.
\end{code}


\subsubsection{Separating the hour, minute, and second}

\begin{code}{key:clock/hour-minute-separator}
  \begin{syntax}
    \key{clock/hour-minute-separator=\meta{token list}}
  \end{syntax}
  Sets the separator between \meta{hour} and \meta{minute} to
  \meta{token list}.
  The initial value is \key{=:}.
\end{code}

\begin{code}{key:clock/minute-second-separator}
  \begin{syntax}
    \key{clock/minute-second-separator=\meta{token list}}
  \end{syntax}
  Sets the separator between \meta{minute} and \meta{second} to
  \meta{token list}.
  The initial value is \key{=:}.
\end{code}

\begin{code}{key:clock/clock-separator}
  \begin{syntax}
    \key{clock/clock-separator=\meta{token list}}
  \end{syntax}
  Meta key that sets the keys \key{clock/hour-minute-separator} and
  \key{clock/minute-second-separator} to \meta{token list}.
\end{code}

\subsection{\cs{calendardate}}
\label{subsec:date}

This subsection documents the keys that modify the behavior of
\cs{calendardate}.

\begin{code}{key:date}
  \begin{syntax}
    \key{date=\meta{key-value list}}
  \end{syntax}
  Meta key that sets the keys in \meta{key-value list} in path |date|.
\end{code}

\begin{code}{key:date/star}
  \begin{syntax}
    \key{date/star=\meta{key-value list}}
  \end{syntax}
  When adding the optional star in \cs{calendardate}, the keys in
  \meta{key-value list} in path |date| will be set for only that particular
  usage of \cs{calendardate}.
  The initial value is \meta{empty}.
\end{code}

\begin{code}{key:date/order}
  \begin{syntax}
    \key{date/order=\choices{month-day-year,day-month-year,year-month-day}}
  \end{syntax}
  Choice key that sets the order in which the year, month, and day are printed.
  The initial value is \key{=month-day-year}.
\end{code}

\begin{code}
    {
      key:date/month/year-month-day,
      key:date/month/year-month,
      key:date/month/month-day,
      key:date/month
    }
  \begin{syntax}
    \key{date/month/year-month-day=\meta{choice}}
    \key{date/month/year-month=\meta{choice}}
    \key{date/month/month-day=\meta{choice}}
    \key{date/month=\meta{choice}}
  \end{syntax}
  \listheading{\meta{choice} must match any of the following:}
  \begin{itemize}
    \item \key{=long}
    \item \key{=abbreviated} (Jan., Feb., Aug., Sept., Oct., Nov., Dec.)
    \item \key{=three-letter}
    \item \key{=number}
    \item \key{=zero-padded-number}
  \end{itemize}
  The choice key \key{date/month/year-month-day} sets the month format when
  printing the year, month, and day.
  The initial value is \key{=abbreviated}.
  The choice key \key{date/month/year-month} sets the month format when
  printing only the year and month.
  The initial value is \key{=long}.
  The choice key \key{date/month/month-day} sets the month format when
  printing only the month and day.
  The initial value is \key{=abbreviated}.
  The meta key \key{date/month} sets the aforementioned keys to \meta{choice}.

  If the next token after \cs{calendardate} is a period, the abbreviated month
  will not print an extra period.
  When the abbreviated month prints the period, it is followed by \cs{@}
  because the sentence does not end here.
\end{code}

\begin{code}{key:date/day}
  \begin{syntax}
    \key{date/day=\choices{number,zero-padded-number}}
  \end{syntax}
  Choice key that sets the format in which the day is printed.
  The initial value is \key{=number}.
\end{code}

\subsubsection{Separating the year, month, and day}

\begin{code}{key:date/year-month-separator}
  \begin{syntax}
    \key{date/year-month-separator=\meta{token list}}
  \end{syntax}
  Sets the separator between \meta{year} and \meta{month} to \meta{token list}.
  The initial value is \key{= }.
\end{code}

\begin{code}{key:date/month-day-separator}
  \begin{syntax}
    \key{date/month-day-separator=\meta{token list}}
  \end{syntax}
  Sets the separator between \meta{month} and \meta{day} to \meta{token list}.
  The initial value is \key{= }.
\end{code}

\begin{code}{key:date/year-day-separator}
  \begin{syntax}
    \key{date/year-day-separator=\meta{token list}}
  \end{syntax}
  Sets the separator between \meta{year} and \meta{day} to \meta{token list}.
  The initial value is \key{=, }.
\end{code}

\begin{code}{key:date/date-separator}
  \begin{syntax}
    \key{date/date-separator=\meta{token list}}
  \end{syntax}
  Meta key that sets the keys \key{date/year-month-separator},
  \key{date/month-day-separator}, and \key{date/year-day-separator} to
  \meta{token list}.
\end{code}

\subsubsection{Printing the days}

\begin{code}{key:date/days-separator}
  \begin{syntax}
    \key{date/days-separator=\meta{token list}}
  \end{syntax}
  Sets the separator between each item in \meta{day(s)} to \meta{token list}.
  The initial value is \key{=, }.
\end{code}

\begin{code}{key:date/days-pair-separator}
  \begin{syntax}
    \key{date/days-pair-separator=\meta{token list}}
  \end{syntax}
  Sets the separator between each item in \meta{day(s)} to \meta{token list}
  when \meta{day(s)} contains exactly two items.
  The initial value is \key{= and }.
\end{code}

\begin{code}{key:date/days-final-separator}
  \begin{syntax}
    \key{date/days-final-separator=\meta{token list}}
  \end{syntax}
  Sets the separator between the last two items in \meta{day(s)} to
  \meta{token list} when \meta{day(s)} contains three or more items.
  The initial value is \key{=, and }.
\end{code}

\begin{code}{key:date/days-range-separator}
  \begin{syntax}
    \key{date/days-range-separator=\meta{token list}}
  \end{syntax}
  Sets the separator between \meta{day_1} and \meta{day_2} in a day range to
  \meta{token list}.
  The initial value is \key{= to }.
\end{code}

\subsubsection{Printing the months}

\begin{code}{key:date/months-separator}
  \begin{syntax}
    \key{date/months-separator=\meta{token list}}
  \end{syntax}
  Sets the separator between each item in \meta{month(s)} to \meta{token list}.
  The initial value is \key{=, }.
\end{code}

\begin{code}{key:date/months-pair-separator}
  \begin{syntax}
    \key{date/months-pair-separator=\meta{token list}}
  \end{syntax}
  Sets the separator between each item in \meta{month(s)} to \meta{token list}
  when \meta{month(s)} contains exactly two items.
  The initial value is \key{= and }.
\end{code}

\begin{code}{key:date/months-final-separator}
  \begin{syntax}
    \key{date/months-final-separator=\meta{token list}}
  \end{syntax}
  Sets the separator between the last two items in \meta{month(s)} to
  \meta{token list} when \meta{month(s)} contains three or more items.
  The initial value is \key{=, and }.
\end{code}

\begin{code}{key:date/months-range-separator}
  \begin{syntax}
    \key{date/months-range-separator=\meta{token list}}
  \end{syntax}
  Sets the separator between \meta{month_1} and \meta{month_2} in a month range
  to \meta{token list}.
  The initial value is \key{= to }.
\end{code}


\section{References}
\label{sec:references}

This package uses some ideas from the \pkg{datetime2} and \pkg{siunitx}
packages.
The \pkg{clockdate} package keys
\begin{itemize}
  \item \key{clock/hour-minute-separator}
  \item \key{clock/minute-second-separator}
  \item \key{clock/clock-separator}
\end{itemize}
resemble the following \pkg{datetime2} package options:
\begin{itemize}
  \item \key{hourminsep}
  \item \key{minsecsep}
  \item \key{timesep}
\end{itemize}

The \pkg{clockdate} package keys
\begin{itemize}
  \item \key{date/year-month-separator}
  \item \key{date/month-day-separator}
  \item \key{date/year-day-separator}
  \item \key{date/date-separator}
\end{itemize}
resemble the following \pkg{datetime2} package options:
\begin{itemize}
  \item \key{yearmonthsep}
  \item \key{monthdaysep}
  \item \key{dayyearsep}
  \item \key{datesep}
\end{itemize}

The \pkg{clockdate} package keys
\begin{itemize}
  \item \key{date/days-final-separator}
  \item \key{date/days-pair-separator}
  \item \key{date/days-separator}
  \item \key{date/months-final-separator}
  \item \key{date/months-pair-separator}
  \item \key{date/months-separator}
\end{itemize}
use a similar naming convention to the following \pkg{siunitx} package control
options:
\begin{itemize}
  \item \key{list-final-separator}
  \item \key{list-pair-separator}
  \item \key{list-separator}
\end{itemize}


\end{document}
