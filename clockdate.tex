\documentclass{beery}


\ExplSyntaxOn
% \debug_on:n { all }
\usepackage[overwrite-date]{clockdate}
\clockdatesetup
  {
    date / star =
      {
          days-final-separator   = { ,~or~ }
        , days-pair-separator    = {  ~or~ }
        , months-final-separator = { ,~or~ }
        , months-pair-separator  = {  ~or~ }
      }
  }
\ExplSyntaxOff


\begin{document}


\section*{\stepfontsize{1}The \pkg{clockdate} package}

Provides \cs{clock} and \cs{calendardate}.

Oliver Beery

Version 0.0.0\quad\date{2025-9-22}


\section{Introduction}
\label{sec:introduction}

\subsection{About}

This package provides \cs{clock} and \cs{calendardate}.
\cs{clock} prints the time of day.
\cs{calendardate} prints the calendar date.

\subsection{Loading the package}
\label{subsec:loading}

\listheading{Requirements:}
\begin{itemize}
  \item \LaTeXe{} version 2023-11-01 or newer
  \item \pkg{l3kernel} version 2023-11-01 or newer
\end{itemize}

\pkg{clockdate} does not load or require any other packages.

\subsection{Syntax}
\label{subsec:syntax}

This documentation uses the syntax \meta{int expr}.
This syntax has the same meaning as the argument to \cs{inteval}, which is documented in \pkg{usrguide}.


\section{Package options}
\label{sec:packageoptions}

This section documents the package options provided by the \pkg{clockdate} package.

\begin{variable}{overwrite-date}
  \begin{syntax}
    \key{overwrite-date}
  \end{syntax}
  Declares a new document command \cs{date} that is equivalent to \cs{calendardate}.
  This new definition of \cs{date} overwrites its previous definition.
  By default, \cs{date} sets the document\rq s date which is used in \cs{maketitle}.
  Use this package option only if your document does not rely on the old definition of \cs{date}.
\end{variable}


\section{Commands}
\label{sec:commands}

% This package defines some commands that take an \meta{int expr}.
% This syntax has the same meaning as the argument to \cs{inteval}, which is documented in \pkg{usrguide}.

\begin{function}{\clockdatesetup}
  \begin{syntax}
    \cs{clockdatesetup} \sarg{} \marg{key-value list}
  \end{syntax}
  Sets and processes the \pkg{clockdate} package keys (\S\ref{sec:keys}) in \meta{key-value list}.
  Adding the optional star first resets the \pkg{clockdate} package keys to their initial values.
  Can be used mid-document.
  The scope of the effect is local to the current group.
\end{function}

\begin{function}{\clock}
  \begin{syntax}
    \cs{clock} \sarg{} \oarg{key-value list} \marg{token list}
  \end{syntax}
  Prints the time of day or a time of day range.
  \meta{token list} must take either of the following forms:
  \begin{itemize}
    \item \meta{clock}
    \item \meta{clock_1}|--|\meta{clock_2}
  \end{itemize}

  \meta{clock} must take any of the following forms:
  \begin{itemize}
    \item \meta{hour}|:|\meta{minute}|:|\meta{second}
    \item \meta{hour}|:|\meta{minute}
    \item \meta{hour}
  \end{itemize}

  \meta{hour}, \meta{minute}, and \meta{second} are evaluated as an \meta{int expr}.
  \meta{hour} must evaluate to an integer from \num{0} to \num{24}.
  \meta{minute} and \meta{second} must evaluate to an integer from \num{0} to \num{59}.

  If the optional argument is used, \meta{key-value list} sets the keys in path \key{clock/} for only that particular usage of \cs{clock}.
  For details on adding the optional star, see the key \key{clock/star} (\S\ref{subsec:clock}).
\end{function}

\begin{function}{\calendardate}
  \begin{syntax}
    \cs{calendardate} \sarg{} \oarg{key-value list} \marg{token list}
  \end{syntax}
  \listheading
    {
      Prints the calendar date.
      \meta{token list} must take any of the following forms:
    }
  \begin{itemize}
    \item \meta{year}|-|\meta{month}|-|\meta{day(s)}
    \item \meta{month}|-|\meta{day(s)}
    \item \meta{year}|-|\meta{month(s)}
  \end{itemize}

  \meta{year} and \meta{month} are evaluated as an \meta{int expr}.
  \meta{year} must evaluate to an integer from \num{1000} to \num{9999}.
  \meta{month} must evaluate to an integer from \num{1} to \num{12}.

  \listheading{\meta{day(s)} must take any of the following forms:}
  \begin{itemize}
    \item a single day \meta{day}
    \item a day range \meta{day_1}|--|\meta{day_2}
    \item a comma-separated list of items where each item must be either a single day \meta{day} or a day range \meta{day_1}|--|\meta{day_2}
  \end{itemize}
  Each \meta{day} is evaluated as an \meta{int expr} and must evaluate to a valid calendar date.
  The calendar date \calendardate{2-29} is always valid if \meta{year} is omitted.

  \listheading{\meta{month(s)} must take any of the following forms:}
  \begin{itemize}
    \item a single month \meta{month}
    \item a month range \meta{month_1}|--|\meta{month_2}
    \item a comma-separated list of items where each item must be either a single month \meta{month} or a month range \meta{month_1}|--|\meta{month_2}
  \end{itemize}

  If the optional argument is used, \meta{key-value list} sets the keys in path \key{date/} for only that particular usage of \cs{calendardate}.
  For details on adding the optional star, see the key \key{date/star} (\S\ref{subsec:date}).
\end{function}


\section{Keys}
\label{sec:keys}

This section documents the keys provided by the \pkg{clockdate} package.
Set the package keys using \cs{clockdatesetup}\marg{key-value list} (\S\ref{sec:commands}).


\subsection{\cs{clock}}
\label{subsec:clock}

This subsection documents the keys that modify the behavior of \cs{clock}.

\begin{variable}{clock}
  \begin{syntax}
    \key{clock=\meta{key-value list}}
  \end{syntax}
  Meta key that sets the keys in \meta{key-value list} in path \key{clock/}.
\end{variable}

\begin{variable}{clock/star}
  \begin{syntax}
    \key{clock/star=\meta{key-value list}}
  \end{syntax}
  When adding the optional star in \cs{clock}, the keys in \meta{key-value list} in path \key{clock/} will be set for only that particular usage of \cs{clock}.
  This key is initially not set.
\end{variable}

\begin{variable}{clock/hour}
  \begin{syntax}
    \key{clock/hour=\choices{12,24}}
  \end{syntax}
  Choice key that sets whether to print the hour in 12-hour or 24-hour format.
  In 24-hour format, the am/pm is always omitted.
  The initial value is \key{=12}.
\end{variable}

\begin{variable}{clock/hour-leading-zero}
  \begin{syntax}
    \key{clock/hour-leading-zero=\choices{true,false}}
  \end{syntax}
  Boolean key that sets whether to print the hour with a leading zero.
  The initial value is \key{=false}.
\end{variable}

\begin{variable}{clock/ampm}
  \begin{syntax}
    \key{clock/ampm=\meta{choice}}
  \end{syntax}
  \listheading
    {
      Choice key that sets the am/pm format.
      \meta{choice} must match any of the following:
    }
  \begin{itemize}
    \item \key{=lowercase-with-periods} (a.m./p.m.)
    \item \key{=lowercase} (am/pm)
    \item \key{=uppercase-with-periods} (A.M./P.M.)
    \item \key{=uppercase} (AM/PM)
    \item \key{=small-caps-with-periods} (\textsc{a}.\textsc{m}./\textsc{p}.\textsc{m}.)
    \item \key{=small-caps} (\textsc{am}/\textsc{pm})
    \item \key{=none}
  \end{itemize}
  The initial value is \key{=lowercase-with-periods}.

  If the next token after \cs{clock} is a period, the am/pm will not print an extra period.
  When the am/pm prints the period, it is followed by \cs{@} because the sentence does not end here.
  % \begin{itemize}
  %   \item
  %   With option \key{=lowercase-with-periods}, the am/pm is displayed as a.m.} and p.m. where the period after the \enquote{m is printed only if the next token is not a period.
  %   This prevents two consecutive periods from being mistakenly printed.
  %   When printing the period after the m}, the period is followed by \cs{@.
  %   With \cs{nonfrenchspacing}, this corrects the interword spacing because this period would not mark the end of a sentence.
  %   \item
  %   With option \key{=lowercase}, the am/pm is displayed as am} or \enquote{pm.
  %   \item
  %   With option \key{=uppercase}, the am/pm is displayed as AM} or PM followed by \cs{@.
  %   With \cs{nonfrenchspacing}, the inserted \cs{@} corrects the interword spacing after the capital letters.
  %   \item
  %   Option \key{=none} omits the am/pm.
  % \end{itemize}
\end{variable}

\begin{variable}{clock/ampm-omit-first}
  \begin{syntax}
    \key{clock/ampm-omit-first=\choices{true,false}}
  \end{syntax}
  Boolean key that sets whether \meta{clock_1} prints the am/pm in a time of day range if \meta{clock_1} and \meta{clock_2} would either both display am or both display pm.
  The initial value is \key{=true}.
\end{variable}

\begin{variable}
  {
    clock/hour-minute-separator,
    clock/minute-second-separator,
    clock/clock-separator
  }
  \begin{syntax}
    \key{clock/hour-minute-separator=\meta{token list}}
    \key{clock/minute-second-separator=\meta{token list}}
    \key{clock/clock-separator=\meta{token list}}
  \end{syntax}
  The key \key{clock/hour-minute-separator} sets the separator between \meta{hour} and \meta{minute} to \meta{token list}.
  The initial value is \key{=:}.
  The key \key{clock/minute-second-separator} sets the separator between \meta{minute} and \meta{second} to \meta{token list}.
  The initial value is \key{=:}.
  The meta key \key{clock/clock-separator} sets the aforementioned keys to \meta{token list}.
\end{variable}

% \begin{variable}{clock/ampm-separator}
%   \begin{syntax}
%     clock/ampm-separator=\meta{token list}
%   \end{syntax}
%   Sets the separator before the am/pm.
%   % With option \key{=space}, the inserted space is non-breaking if only a single-digit hour is printed.
%   % When printing a time of day range, the space before the second am/pm is non-breaking if the second clock contains only a single digit hour and the first am/pm is printed.
% \end{variable}

% \begin{variable}{clock/ampm-separator}
%   \begin{syntax}
%     clock/ampm-separator=\choices{space,thinspace,none}
%   \end{syntax}
%   Choice key that sets the space before the am/pm.
%   % With option \key{=space}, the inserted space is non-breaking if only a single-digit hour is printed.
%   % When printing a time of day range, the space before the second am/pm is non-breaking if the second clock contains only a single digit hour and the first am/pm is printed.
% \end{variable}

\begin{variable}{clock/ampm-separator}
  \begin{syntax}
    \key{clock/ampm-separator=\meta{token list}}
  \end{syntax}
  Sets the separator before the am/pm to \meta{token list}.
  The initial value is \key{=\textvisiblespace}.
\end{variable}

% \begin{variable}{clock/range-separator}
%   \begin{syntax}
%     clock/range-separator=\choices{to,endash}
%   \end{syntax}
%   Choice key that sets the separator between \meta{clock_1} and \meta{clock_2} in a time of day range.
%   Option \key{=to} sets the separator to\key{=,\textvisiblespace to\textvisiblespace}}.
%   Option \key{=endash} sets the separator to an en dash.

%   A line break is always disallowed before the en dash.
%   A line break is disallowed after the en dash in either case:
%   \begin{itemize}
%     \item
%     \meta{clock_1} prints only a single-digit hour and omits the am/pm.
%     \item
%     \meta{clock_2} prints only the hour and either omits the am/pm or the value of the key \key{clock/ampm-separator} is \key{=space}.
%   \end{itemize}
% \end{variable}
\begin{variable}{clock/range-separator}
  \begin{syntax}
    \key{clock/range-separator=\meta{token list}}
  \end{syntax}
  Sets the separator between \meta{clock_1} and \meta{clock_2} in a time of day range to \meta{token list}.
  The initial value is \key{=\textvisiblespace to\textvisiblespace}.
\end{variable}


\subsection{\cs{calendardate}}
\label{subsec:date}

This subsection documents the keys that modify the behavior of \cs{calendardate}.

\begin{variable}{date}
  \begin{syntax}
    \key{date=\meta{key-value list}}
  \end{syntax}
  Meta key that sets the keys in \meta{key-value list} in path \key{date/}.
\end{variable}

\begin{variable}{date/star}
  \begin{syntax}
    \key{date/star=\meta{key-value list}}
  \end{syntax}
  When adding the optional star in \cs{calendardate}, the keys in \meta{key-value list} in path \key{date/} will be set for only that particular usage of \cs{calendardate}.
  This key is initially not set.
\end{variable}

\begin{variable}{date/order}
  \begin{syntax}
    \key{date/order=\choices{month-day-year,day-month-year,year-month-day}}
  \end{syntax}
  Choice key that sets the order in which the year, month, and day are printed.
  The initial value is \key{=month-day-year}.
\end{variable}

\begin{variable}
    {
      date/month/year-month-day,
      date/month/year-month,
      date/month/month-day,
      date/month
    }
  \begin{syntax}
    \key{date/month/year-month-day=\meta{choice}}
    \key{date/month/year-month=\meta{choice}}
    \key{date/month/month-day=\meta{choice}}
    \key{date/month=\meta{choice}}
  \end{syntax}
  \listheading{\meta{choice} must match any of the following:}
  \begin{itemize}
    \item \key{=long}
    \item \key{=abbreviated} (Jan., Feb., Aug., Sept., Oct., Nov., Dec.)
    \item \key{=three-letter}
    \item \key{=number}
    \item \key{=zero-padded-number}
  \end{itemize}
  The choice key \key{date/month/year-month-day} sets the month format when printing the year, month, and day.
  The initial value is \key{=abbreviated}.
  The choice key \key{date/month/year-month} sets the month format when printing only the year and month.
  The initial value is \key{=long}.
  The choice key \key{date/month/month-day} sets the month format when printing only the month and day.
  The initial value is \key{=abbreviated}.
  The meta key \key{date/month} sets the aforementioned keys to \meta{choice}.

  If the next token after \cs{calendardate} is a period, the abbreviated month will not print an extra period.
  When the abbreviated month prints the period, it is followed by \cs{@} because the sentence does not end here.
\end{variable}

%   % With option \key{=long}, the name of each month is fully spelled out.
%   % With option \key{=abbreviated}, some months are abbreviated: Jan., Feb., Aug., Sept., Oct., Nov., Dec.
%   % The period in each abbreviation is printed only if the next token is not a period.
%   % This prevents two consecutive periods from being mistakenly printed.
%   % When printing the period, the period is followed by \cs{@}.
%   % With \cs{nonfrenchspacing}, this corrects the interword spacing because this period would not mark the end of a sentence.
%   % With option \key{=three-letter}, each month is shortened to its first three letters.

\begin{variable}
    {
      date/year-month-separator,
      date/month-day-separator,
      date/year-day-separator,
      date/date-separator
    }
  \begin{syntax}
    \key{date/year-month-separator=\meta{token list}}
    \key{date/month-day-separator=\meta{token list}}
    \key{date/year-day-separator=\meta{token list}}
    \key{date/date-separator=\meta{token list}}
  \end{syntax}
  The key \key{date/year-month-separator} sets the separator between \meta{year} and \meta{month} to \meta{token list}.
  The initial value is \key{=\textvisiblespace}.
  The key \key{date/month-day-separator} sets the separator between \meta{month} and \meta{day} to \meta{token list}.
  The initial value is \key{=\textvisiblespace}.
  The key \key{date/year-day-separator} sets the separator between \meta{year} and \meta{day} to \meta{token list}.
  The initial value is \key{=,\textvisiblespace}.
  The meta key \key{date/date-separator} sets the aforementioned keys to \meta{token list}.
\end{variable}

\begin{variable}
  {
    date/days-separator,
    date/days-pair-separator,
    date/days-final-separator
  }
  \begin{syntax}
    \key{date/days-separator=\meta{token list}}
    \key{date/days-pair-separator=\meta{token list}}
    \key{date/days-final-separator=\meta{token list}}
  \end{syntax}
  The key \key{date/days-separator} sets the separator between each item in \meta{day(s)} to \meta{token list}.
  The initial value is \key{=,\textvisiblespace}.
  The key \key{date/days-pair-separator} sets the separator between each item in \meta{day(s)} to \meta{token list} when \meta{day(s)} contains exactly two items.
  The initial value is \key{=\textvisiblespace and\textvisiblespace}.
  The key \key{date/days-final-separator} sets the separator between the last two items in \meta{day(s)} to \meta{token list} when \meta{day(s)} contains three or more items.
  The initial value is \key{=,\textvisiblespace and\textvisiblespace}.
\end{variable}

\begin{variable}{date/days-range-separator}
  \begin{syntax}
    \key{date/days-range-separator=\meta{token list}}
  \end{syntax}
  Sets the separator between \meta{day_1} and \meta{day_2} in a day range to \meta{token list}.
  The initial value is \key{=\textvisiblespace to\textvisiblespace}.
\end{variable}

\begin{variable}
  {
    date/months-separator,
    date/months-pair-separator,
    date/months-final-separator
  }
  \begin{syntax}
    \key{date/months-separator=\meta{token list}}
    \key{date/months-pair-separator=\meta{token list}}
    \key{date/months-final-separator=\meta{token list}}
  \end{syntax}
  The key \key{date/months-separator} sets the separator between each item in \meta{month(s)} to \meta{token list}.
  The initial value is \key{=,\textvisiblespace}.
  The key \key{date/months-pair-separator} sets the separator between each item in \meta{month(s)} to \meta{token list} when \meta{month(s)} contains exactly two items.
  The initial value is \key{=\textvisiblespace and\textvisiblespace}.
  The key \key{date/months-final-separator} sets the separator between the last two items in \meta{month(s)} to \meta{token list} when \meta{month(s)} contains three or more items.
  The initial value is \key{=,\textvisiblespace and\textvisiblespace}.
\end{variable}

\begin{variable}{date/months-range-separator}
  \begin{syntax}
    \key{date/months-range-separator=\meta{token list}}
  \end{syntax}
  Sets the separator between \meta{month_1} and \meta{month_2} in a month range to \meta{token list}.
  The initial value is \key{=\textvisiblespace to\textvisiblespace}.
\end{variable}


\section{References}
\label{sec:references}

This package uses some ideas from the \pkg{datetime2} and \pkg{siunitx} packages.
The \pkg{clockdate} package keys
\begin{itemize}
  \item \key{clock/hour-minute-separator}
  \item \key{clock/minute-hour-separator}
  \item \key{clock/clock-separator}
\end{itemize}
resemble the following \pkg{datetime2} package options:
\begin{itemize}
  \item \key{hourminsep}
  \item \key{minsecsep}
  \item \key{timesep}
\end{itemize}

The \pkg{clockdate} package keys
\begin{itemize}
  \item \key{date/year-month-separator}
  \item \key{date/month-day-separator}
  \item \key{date/year-day-separator}
  \item \key{date/date-separator}
\end{itemize}
resemble the following \pkg{datetime2} package options:
\begin{itemize}
  \item \key{yearmonthsep}
  \item \key{monthdaysep}
  \item \key{dayyearsep}
  \item \key{datesep}
\end{itemize}

The \pkg{clockdate} package keys
\begin{itemize}
  \item \key{date/days-final-separator}
  \item \key{date/days-pair-separator}
  \item \key{date/days-separator}
  \item \key{date/months-final-separator}
  \item \key{date/months-pair-separator}
  \item \key{date/months-separator}
\end{itemize}
use a similar naming convention to the following \pkg{siunitx} package control options:
\begin{itemize}
  \item \key{list-final-separator}
  \item \key{list-pair-separator}
  \item \key{list-separator}
\end{itemize}


\end{document}